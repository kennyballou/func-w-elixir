\documentclass[english]{beamer}
\usetheme{Berlin}
\usecolortheme[light,accent=blue]{solarized}
\usepackage{amsmath}
\usepackage{unicode-math}
\usepackage{fontspec}
\usepackage{xunicode}
\usepackage{xltxtra}
\setmainfont{DejaVuSans}
\usepackage{booktabs}
\usepackage{lmodern}
\usepackage{listings}
\usepackage{color}
\usepackage{babel}
\usepackage{tikz}
\usetikzlibrary{trees, shapes.misc, arrows}
\usepackage{pgfkeys}
\usepackage{graphicx}
\graphicspath{{./images/}}
\setbeamertemplate{headline}{}

\lstset{%
    basicstyle=\footnotesize\ttfamily,
    breakatwhitespace=false,
    breaklines=true,
    captionpos=b,
    frame=signle,
    keepspaces=true,
    columns=flexible,
    language=Ruby,
    numbers=none,
    numbersep=5pt,
    numberstyle=\tiny\color{solarizedBase00},
    showspaces=false,
    showstringspaces=false,
    stepnumber=1,
    showtabs=false,
    stringstyle=\color{solarizedMagenta},
    keywordstyle=\color{solarizedCyan},
    commentstyle=\color{solarizedGreen},
    tabsize=2,
    extendedchars=true
}

\title{Learning Functional Programming with Elixir}
\subtitle{A Short Guide Through Functional Programming}
\author[Ballou]{Kenny Ballou}
\institute[zData]{%
    \inst{}%
    zData, Inc.
}

\AtBeginSection[]{%
    \begin{frame}
    \tableofcontents[
        currentsection,
        sectionstyle=show/shaded,
        subsectionstyle=show/show/hide]
    \end{frame}
}

\begin{document}
% TikZ Styles
\tikzstyle{every node}=[%
    fill=solarizedBase02,
    draw=solarizedBase01,
    thick,
    rounded corners,
    anchor=north,
    sibling distance=6cm]
\tikzstyle{edge from parent}=[%
    solarizedBase00,
    -o,
    thick,
    draw]

%\tikzstyle{edge from parent path}=[%
%    \tikzparentnode.east |- \tikzchildnode.south]

\begin{frame}[label=titleslide]
\titlepage{}
\end{frame}

\begin{frame}
\tableofcontents[subsectionstyle=hide]
\end{frame}

\begin{frame}
\frametitle{Who am I?}
\begin{itemize}
\item{Hacker}
\item{Developer (read gardener)}
\item{Mathematician}
\item{Student}
\end{itemize}

\end{frame}

\section{Introduction}

\begin{frame}
\begin{figure}
\includegraphics[scale=0.45]{xkcd_functional.png}
\caption{``Functional programming combines the flexibility and power of
abstract mathematics with the intuitive clarity of abstract mathematics.''}
\end{figure}
XKCD on Functional Programming\cite{website:xkcd_functional}
\end{frame}

\begin{frame}
\frametitle{What is Functional Programming?}
\begin{itemize}
\item<1->{Functional programming is a paradigm} %you might say pattern
\item<2->{Prefers ``mathematical'' functions}
\item<3->{Uses (mostly) immutable data} %depending on "purity"
\item<4->{Everything is an expression}
\end{itemize}
\end{frame}

\begin{frame}
\frametitle{What is Elixir? Erlang/OTP?}
\begin{itemize}
\item{Elixir is a new functional, dynamically typed language}
\item{Elixir's design is heavily influenced by Erlang's design}
\item{Erlang is an old functional, dynamically typed language, first appeared
in 1986}
\item{Built around concurrency, fault-tolerance, and high-availability}
\item{Elixir compiles to BEAM (Erlang) bytecode}
\item{Elixir ``looks'' like Ruby} %but don't even try to write ruby...
\end{itemize}
\end{frame}

\section{Elixir Basics}

\begin{frame}[fragile]
\frametitle{Interactive Elixir}
\framesubtitle{\texttt{iex}}
\lstinputlisting{code/1/iex}
\end{frame}

\begin{frame}[fragile]
\frametitle{Basic Types}
\framesubtitle{Numerals}
\lstinputlisting[lastline=16]{code/1/numbers}
\end{frame}

\begin{frame}[fragile]
\frametitle{Basic types}
\framesubtitle{Numeral Operators}
\lstinputlisting[firstline=18]{code/1/numbers}
\end{frame}

\begin{frame}[fragile]
\frametitle{Basic Types}
\framesubtitle{Atoms}
\lstinputlisting[lastline=10]{code/1/atoms}
\end{frame}

\begin{frame}[fragile]
\frametitle{Basic Types}
\framesubtitle{Atom Functions}
\lstinputlisting[firstline=12]{code/1/atoms}
\end{frame}

\begin{frame}[fragile]
\frametitle{Basic Types}
\framesubtitle{Booleans}
\lstinputlisting[lastline=6]{code/1/booleans}
\end{frame}

\begin{frame}[fragile]
\frametitle{Basic Types}
\framesubtitle{Boolean Operators}
\lstinputlisting[firstline=8]{code/1/booleans}
\end{frame}

\begin{frame}[fragile]
\frametitle{Basic Types}
\framesubtitle{Lists}
\lstinputlisting[lastline=12]{code/1/lists}
\end{frame}

\begin{frame}[fragile]
\frametitle{Basic Types}
\framesubtitle{List Operations}
\lstinputlisting[firstline=14]{code/1/lists}
\end{frame}

\begin{frame}[fragile]
\frametitle{Basic Types}
\framesubtitle{Tuples}
\lstinputlisting[lastline=6]{code/1/tuples}
\end{frame}

\begin{frame}[fragile]
\frametitle{Basic Types}
\framesubtitle{Tuple Operations}
\lstinputlisting[firstline=8]{code/1/tuples}
\end{frame}

\begin{frame}[fragile]
\frametitle{Basic Types}
\framesubtitle{Binaries}
\lstinputlisting[lastline=8]{code/1/binaries}
\end{frame}

\begin{frame}[fragile]
\frametitle{Basic Types}
\framesubtitle{Binary Operations}
\lstinputlisting[firstline=10]{code/1/binaries}
\end{frame}

\begin{frame}[fragile]
\frametitle{Basic Types}
\framesubtitle{Maps}
\lstinputlisting[lastline=10]{code/1/maps}
\end{frame}

\begin{frame}[fragile]
\frametitle{Basic Types}
\framesubtitle{Map Operations}
\lstinputlisting[firstline=12]{code/1/maps}
\end{frame}

\begin{frame}[fragile]
\frametitle{Basic Types}
\framesubtitle{Functions}
\lstinputlisting{code/1/functions}
\end{frame}

\begin{frame}
\frametitle{Dispelling Assignment}
\framesubtitle{There is no spoon}
\begin{itemize}
\item{\texttt{=} does \textbf{not} mean \textit{assign}}
\item{\texttt{x = 1} is not \textit{assign 1 to \texttt{x}}}
\item{\texttt{=} is a \textit{match} operator}
\begin{itemize}
\item{\texttt{=} is a constraint-solving operator}
\end{itemize}
\end{itemize}
\end{frame}

\begin{frame}[fragile]
\frametitle{Pattern Matching}
\lstinputlisting[lastline=8]{code/1/patterns}
\end{frame}

\begin{frame}[fragile]
\frametitle{Pattern Matching}
\lstinputlisting[firstline=10,lastline=28]{code/1/patterns}
\end{frame}

\begin{frame}[fragile]
\frametitle{Pattern Matching}
\lstinputlisting[firstline=30,lastline=41]{code/1/patterns}
\end{frame}

\begin{frame}
\frametitle{Pattern Matching}
\framesubtitle{Brief Introduction to IEEE-754}
\begin{itemize}
\item{64-bit floating point (doubles) numbers are represented using IEEE-754}
\item{32-bit (single/\texttt{float}) and 128-bit (quadurpals) are similarly
represented with varying number of bits for each component}
\item{There are four main components}
\begin{itemize}
\item{sign, ±, 1 bit}
\item{exponent, 11 bits}
\item{fraction (mantissa), 52 bits}
\item{bias, built-in, typcially $1023$ for doubles}
\end{itemize}
\end{itemize}
\end{frame}

\begin{frame}
\frametitle{Pattern Matching}
\framesubtitle{Brief Introduction to IEEE-754}
To convert from binary bits to a ``float'', we can use the following formula:
$$
{-1}^{\text{sign}} \cdot{}
\frac{1 + \text{mantissa}}{2^{52}}
\cdot{} 2^{\text{exponent} - \text{bias}}
$$
\end{frame}

\begin{frame}[fragile]
\frametitle{Pattern Matching}
\lstinputlisting[firstline=43]{code/1/patterns}
\end{frame}

\section{Functional Approach}

\begin{frame}
\frametitle{The Functional Approach}
\framesubtitle{To all the things!}
\begin{itemize}
\item{Less iteration} %sorta
\item{More (Tail) Recursion} %yay!
\item{Performance}
\end{itemize}
\end{frame}

\begin{frame}
\frametitle{Fibonacci}
Because no functional introduction is complete without it.
\begin{align*}
F_n &= F_{n-1} + F_{n-2} \\
F_0 &= 0 \\
F_1 &= 1
\end{align*}
\end{frame}

\begin{frame}[fragile]
\frametitle{Fibonacci}
\lstinputlisting[numbers=left]{code/2/fib_rec.exs}
\end{frame}

\begin{frame}[fragile]
\frametitle{Fibonacci}
\framesubtitle{Iteratively}
\lstinputlisting[numbers=left]{code/2/fib_itr.exs}
\end{frame}

\begin{frame}[fragile]
\frametitle{Fibonacci Performance}
\lstinputlisting[language=bash]{code/2/fib_perf.out}
\end{frame}

\begin{frame}
\frametitle{Quicksort}
\begin{itemize}
\item{Similar to merge sort}
\item{Sort by partitioning}
\item{Has a nice recursive definition} %more yay!
\end{itemize}
\end{frame}

\begin{frame}[fragile]
\frametitle{Quicksort}
\lstinputlisting[numbers=left]{code/2/qs.exs}
\lstinputlisting{code/2/qs.out}
\end{frame}

\begin{frame}
\frametitle{Map-Reduce}
\begin{itemize}
\item<1->{Functional way to process collections}
\item<2->{Can be (partially) pipelined}
\item<3->{Mapping can be lazy}
\item<4->{Map is Reduce} %with a widening accumlator
\item<5->{They are ``folds''}
\end{itemize}
\end{frame}

\begin{frame}[fragile]
\frametitle{Map-Reduce}
\framesubtitle{Implementing our own Map}
\lstinputlisting[numbers=left]{code/2/my_map.exs}
\lstinputlisting{code/2/my_map.out}
\end{frame}

\begin{frame}[fragile]
\frametitle{Map-Reduce}
\framesubtitle{Implementing our own (simple) Reduce}
\lstinputlisting[numbers=left]{code/2/my_red.exs}
\lstinputlisting{code/2/my_red.out}
\end{frame}

\begin{frame}[fragile]
\frametitle{Map-Reduce}
\framesubtitle{Map-Redux}
\lstinputlisting[numbers=left]{code/2/my_map_red.exs}
\end{frame}

\begin{frame}[fragile]
\frametitle{Map-Reduce}
\framesubtitle{Map-Redux}
\lstinputlisting{code/2/my_map_red.out}
\end{frame}

\begin{frame}
\frametitle{Word Counting}
\begin{itemize}
\item{Stream lines out of a file}
\item{Tokenize into words}
\item{Reduce words and print results}
\end{itemize}
\end{frame}

\begin{frame}
\frametitle{Word Counting}
\begin{figure}
\begin{tikzpicture}
    \node (stream) at (0, 0) {Stream};
    \node (tokenize) at (2, 0) {Tokenize};
    \node (reduce) at (4, 0) {Reduce};
    \node (print) at (4, -2) {Print};
    \draw[-o] (stream) -- (tokenize);
    \draw[-o] (tokenize) -- (reduce);
    \draw[-o] (reduce) -- (print);
\end{tikzpicture}
\end{figure}
\end{frame}

\begin{frame}[fragile]
\frametitle{Word Counting}
\framesubtitle{Streaming Data}
\lstinputlisting[numbers=left,firstline=3,lastline=5]{code/2/wc.exs}
\end{frame}

\begin{frame}[fragile]
\frametitle{Word Counting}
\framesubtitle{Tokenizing Words}
\lstinputlisting[numbers=left,firstline=7,lastline=9]{code/2/wc.exs}
\end{frame}

\begin{frame}[fragile]
\frametitle{Word Counting}
\framesubtitle{Reducing Words}
\lstinputlisting[numbers=left,firstline=11,lastline=17]{code/2/wc.exs}
\end{frame}

\begin{frame}[fragile]
\frametitle{Word Counting}
\framesubtitle{Counting Words}
\lstinputlisting[numbers=left,firstline=19,lastline=24]{code/2/wc.exs}
\end{frame}

\begin{frame}[fragile]
\frametitle{Word Counting}
\framesubtitle{Results}
\lstinputlisting{code/2/wc.out}
\end{frame}

\section*{References}
\begin{frame}[allowframebreaks]
\frametitle{References}
\nocite{*}
\renewcommand{\refname}{}
\bibliographystyle{plain}
\bibliography{references}
\end{frame}

\againframe{titleslide}
\end{document}
